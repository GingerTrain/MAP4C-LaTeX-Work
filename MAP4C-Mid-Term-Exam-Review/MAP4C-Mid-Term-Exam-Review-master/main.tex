\documentclass{article}
\usepackage[utf8]{inputenc}
\usepackage{multicol} % Multiple columns
\usepackage{parskip} % Removes indent
\usepackage{gensymb} % Degree symbol
\usepackage{amsmath} % Smoother fractions

\title{MAP4C Review}
\author{Eric Taylor } % Fix title, name, and date to not look like garbage
\date{April 2017}

\DeclareUnicodeCharacter{2220}{\angle} % Declares angle symbol

\begin{document}

\topmargin=0pt % Sets top margin to 0pt
\headheight=0pt % Sets header distance to 0pt
\headsep=0pt % Sets distance between header and body to 0pt
\footskip=0pt % Sets footer to 0pt
\textheight=700pt % Sets height of body to 700pt

\setlength{\parindent}{0pt} % Removes indent

\maketitle \thispagestyle{empty}
%\pagestyle{empty} Removes page number

\textbf{{\huge Unit 1 - Algebra}}

\textbf{\emph{\textsc{1. Expand and simplify}}} \\
\begin{tabular}{l l l l}
    \textbf{a) $(p-4)(p+5)$} & \textbf{b) $(n+8)(n+2)$} & \textbf{c) $(m+3)(m-7)$} 
    & \\
    $= p^2+5p-4p-20$ & = $n^2+2n+8n+16$ & $= m^2-7m+3m-21$ 
    & \\
    $= p^2+p-20$ & $= n^2+10n+16$ & $= m^2-4m-21$
    & \\ \\
    \textbf{d) $(w-4)(w-6)$} & \textbf{e) $(2x-3)(5x-4)$} & \textbf{f) $(y-5)(2y+9)$}
    & \\
    $= w^2-6w-4w+24$ & $= 10x^2-8x-15x+12$ & $= 2y^2+9y-10y-45$
    & \\
    $=w^2-10w+24$ & $=10x^2-23x+12$ & $=2y^2-y-45$
    & \\ \\
    \textbf{g) $(3a+1)(4a+1)$} & \textbf{h) $(7-x)(4+x)$} & \textbf{i) $(x-6)(x+6)$}
    & \\
    $= 12a^2+3a+4a+1$ & $= x^2+4x-7x-28$ & $= x^2+6x-6x-36$
    & \\
    $= 12a^2+7a+1$ & $= x^2-3x-28$ & $= x^2-36$
    & \\ \\
    \textbf{j) $(x+2)^2$} & \textbf{k) $(y+11)(y-11)$} & \textbf{l) $(z-5)^2$}
    & \\
    $= (x+2)(x+2)$ & $= y^2-11y+11y-121$ & $= (z-5)(z-5)$
    & \\
    $= x^2+2x+2x+4$ & $= y^2-121$ & $= z^2-5z-5z+25$
    & \\
    $= x^2+4x+4$ & & $= z^2-10z+25$
    & \\ 
    
\end{tabular}

\textbf{\emph{\textsc{2. Expand and simplify}}} \\
\begin{tabular}{l l l}
    \textbf{a) $4(x-2)(3x-5)$} & \textbf{b) $(4x-3)^2-5(3x^2-5x+7)$}
    & \\
    $= (4x-8)(3x-5)$ & $= (4x-3)(4x-3)-(15x^2-25x+35)$
    & \\
    $= 12x^2-20x-24x+40$ & $= (16x^2-12x-12x+9)-(15x^2-25x+35)$
    & \\
    $= 12x^2-44x+40$ & $= x^2-49x-26$
    & \\ \\
\end{tabular}

\textbf{\emph{\textsc{3. Find an expression for the area of each shape (Refer to handout)}}} \\
\begin{tabular}{l l l}
    \textbf{a) $A= lw$} & \textbf{b) $A= lw + lw$}
    & \\
    $= (2x-5)(x+1)$ & $= (5x)(4x-1)+(4x+3)(2x-1)$
    & \\
\end{tabular}

\newpage

\textbf{\emph{\textsc{4. Factor each of the following completely.}}} \\
\begin{tabular}{l l l l l}
    \textbf{a) $18-12x$} & \textbf{b) $21w^2-28w+35$} & \textbf{c) $24x^2+16x$} & \textbf{d) $15a^3-20a^2+25a$}
    & \\
    $= 3-2x$ & $= 3w^2-4w+5$ & $= 3x+2$ & $= 3a^2-4a+5$
    & \\ \\
    \textbf{e) $20m^2-30m$} & \textbf{f) $27k^3-36k^5$} & \textbf{g) $n^2-144$} & \textbf{h) $81-x^2$}
    & \\
    $= 2m-3$ & $= 3-k^2$ & $= (n-12)(n+12)$ & $= (x-9)(x+9)$
    & \\ \\
    \textbf{i) $5m^2-80$} & \textbf{j) $10x^2-90$} & \textbf{k) $x^2+5x+6$} & \textbf{l) $a^2-a-30$}
    & \\
    $= 2m-3$ & $= 3-k^2$ & $= (n-12)(n+12)$ & $= (x-9)(x+9)$
    & \\ \\
    \textbf{m) $x^2+3x-10$} & \textbf{n) $m^2-9m+20$} & \textbf{o) $x^2+6x-27$} & \textbf{p) $3x^2-6x-105$}
    & \\
    $= (x+5)(x-2)$ & $= (m-5)(m-4)$ & $= (x+9)(x-3)$ & $= 3(x-7)(x+5)$
    & \\ \\
    \textbf{q) $5x^2+17x+6$} & \textbf{r) $5x^2-7x-6$} & \textbf{s) $3x^2+10x+3$} & \textbf{t) $2x^2+9x+4$}
    & \\
    $= (5x+2)(x+3)$ & $= (5x+3)(x-2)$ & $= (3x+1)(x+3)$ & $= (2x+1)(x+4)$
    & \\ \\
\end{tabular}

\textbf{\emph{\textsc{5. Solve each quadratic equation.}}}
\begin{multicols}{3}
    \textbf{a) $x^2-7x+10=0$} \\
    $x^2-5x-2x+10=0$ \\
    $x(x-5)-2(x+5)=0$ \\
    $(x-5)(x-2)=0$ \\
    $x-5=0$ \hspace{0.45cm} $x-2=0$ \\
    $x=5$ \hspace{1cm} $x=2$ \\ \\
    \textbf{d) $x^2-3x=10$} \\
    $x^2-5x+2x-10=0$ \\
    $x(x-5)+2(x-5)=0$ \\
    $(x-5)(x+2)=0$ \\
    $x-5=0$ \hspace{0.45cm}  $x+2=0$ \\
    $x=5$ \hspace{1cm} $x=2$ \\ \\   
    \columnbreak
    
    \textbf{b) $x^2-x-12=0$} \\
    $x^2-4x+3x-12=0$ \\
    $x(x-4)+3(x-4)=0$ \\
    $(x-4)(x+3)=0$ \\
    $x-4=0$ \hspace{0.45cm} $x+3=0$ \\
    $x=4$ \hspace{1cm} $x=-3$ \\ \\
    \textbf{e) $x^2+10x+25=0$} \\ 
    $x^2+5x+5x+25=0$ \\
    $x(x+5)+5(x+5)=0$ \\
    $(x+5)(x+5)=0$ \\
    $x+5=0$ \hspace{0.45cm} $x+5=0$ \\
    $x=-5$ \hspace{1cm} $x=-5$ \\ \\
    \columnbreak
    
    \textbf{c) $x^2-25=0$}\\
    $x^2+5x-5x-25=0$ \\
    $x(x+5)-5(x-5)=0$ \\
    $(x+5)(x-5)=0$ \\
    $x+5=0$ \hspace{0.45cm} $x-5=0$ \\
    $x=-5$ \hspace{1cm} $x=5$ \\ \\
    \textbf{f) $x^2+81=18x$} \\ 
    $x^2-9x-9x+81=0$ \\
    $x(x-9)-9(x+9)=0$ \\
    $(x-9)(x-9)=0$ \\
    $x-9=0$ \hspace{0.45cm} $x-9=0$ \\
    $x=9$ \hspace{1cm} $x=9$ \\ \\
\end{multicols}

\newpage

\textbf{{\huge Unit 2 - Budgets}}

\textbf{\emph{\textsc{1. The Marcella family spends $\$250$ per week on groceries. How much should they budget monthly for their grocery costs? \\
A. $\$1000$ B. $\$1083.33$ C. $\$531.67$ D. $\$500$}}} \\
They should budget $\$1083.33$ monthly for their grocery costs. \\
$\$250*4.345=\$1083.33$ \\

\textbf{\emph{\textsc{2. Which is not a tenant's responsibility? \\
A. Pay rent on time  \\B. Give 60 days written notice before moving out \\C. Pay property tax \\D. Repair any damage he or she causes\\}}}
It is not a tenants responsibility to pay property tax. Payment of property tax is a responsibility of the landlord. \\

\textbf{\emph{\textsc{3. Explain the difference between fixed and variable costs in relation to renting or owning your own home. Include examples.}}} \\
The difference between fixed and variable costs is that fixed costs don't change from period to period, while variable costs can change from period to period. An example of a fixed cost would be rent, as your rent stays the same. An example of a variable cost would be your groceries, as the price can be more or less depending on what you purchase. \\

\textbf{\emph{\textsc{4. Raj is moving to London to attend college. He finds two housing options near the campus. \\
a) What would be the annual rent for each option? \\
b) What additional expense will Raj have if he chooses Option 1? \\
c) Which option would you recommend to Raj? Justify your answer.}}} \\
a)
\begin{tabular}{l l} \\
    Option 1 & Option 2 \\
    $\$350*12=\$4200/yr$ & $\$470*12=\$5640/yr$
\end{tabular} \\ \\
b) Raj will have to pay one-quarter of all utilities as rent as an additional cost if he chooses Option 1. \\ \\
c) I would recommend Option 1 to Raj because it is arguably the cheapest option, assuming the three other students that are living in the house aren't putting the utility cost through the roof. Raj would be in slight control of the rent as he can choose how much utility he uses. \\ \\

\newpage

\textbf{\emph{\textsc{6. a) List two housing expenses that both renters and owners may have to pay. \\
b) List three expenses that only owners would pay.}}} \\ \\
a) Furniture and appliances \\
b) Repairs, upkeep, and property tax

\textbf{\emph{\textsc{8. Determine how much each housing expense will cost for one year.}}}

\begin{tabular}{l l}
    \textbf{a) Weekly mortgage payments of \$346.78}
    & \\
    $\$346.78*52=\$18032.56$
    & \\ \\
    \textbf{b) Monthly phone bill of \$29.95}
    & \\
    $\$29.95*12=\$359.40$
    & \\ \\
    \textbf{c) Monthly rent of \$765}
    & \\
    $\$765*12=\$9180$
    & \\ \\
    \textbf{d) Bi-annual insurance bill of \$427.60}
    & \\
    $\$427.60*2=\$855.20$
    & \\ \\
    \textbf{e) Quarterly water heater rental of \$68.95}
    & \\
    $\$68.95*4=\$275.80$
\end{tabular}

\textbf{\emph{\textsc{9. Ivor estimated his monthly housing costs. Calculate the total cost for one year.}}} \\
\begin{tabular}{|l l l l|}
    \hline
    Rent & monthly & $\$950$ 
    & \\
    Insurance & bi-monthly & $\$35$
    & \\
    Cable/Internet & monthly & $\$95$
    & \\
    Electricity & monthly & $\$45$
    & \\
    Furniture & semi-annually & $\$750$
    & \\
    \hline
\end{tabular}

$= (\$950*12)+(\$35*24)+(\$95*12)+(\$45*12)+(\$750*2)$ \\
$= \$15420$ \\

\newpage

\textbf{{\huge Unit 3 \& 4 - Trigonometry \& \\ Measurement}}

\textbf{\emph{\textsc{1. Use primary trigonometric ratios to determine each measure. \\ (Refer to handout)}}}
\begin{multicols}{2}
    \textbf{a) Side $c$} \\
    $sin(20\degree)=\frac{c}{2}$ \\
    $c=2*sin(20\degree)$ \\
    $c=0.68$ m \\
    \columnbreak
    
    \textbf{b) $\angle{N}$} \\
    $tanN=\frac{4.4}{1.6}$ \\
    $\angle{N}=tan^{-1}(\frac{4.4}{1.6})$ \\
    $\angle{N}=70\degree$ 
\end{multicols}
\textbf{\emph{\textsc{2. Sketch and solve each triangle.}}}

\textbf{a)$\triangle$ABC with $\angle{A}=15\degree$, $\angle{C}=90\degree$, and $c=8$ cm} \\ \\
\begin{tabular}{l l l l}
    $\angle{B}=180\degree-90\degree-15\degree$ & $sin(15\degree)=\frac{a}{8}$ & $sin(75\degree)=\frac{b}{8}$
    & \\
    $\angle{B}= 75\degree$ & $a=8*sin(15\degree)$ & $b=8*sin(75\degree)$
    & \\
    & $a=2.1$ cm & $b=7.7$ cm
\end{tabular}

\textbf{b)$\triangle$CDE with $\angle{E}=90\degree$, $e=14.0$ yards, and $c=9.2$ yards} \\ \\
\begin{tabular}{l l l l}
    $sinC=\frac{9.2}{14}$ & $\angle{D}=180\degree-90\degree-41\degree$ & $sin(49\degree)=\frac{d}{14}$
    & \\
    $\angle{C}=sin^{-1}(\frac{9.2}{14})$ & $\angle{D}=49\degree$ & $d=14*sin(49\degree)$
    & \\
    $\angle{C}=41\degree$ &  & $d=10.6$ yards
\end{tabular}

\textbf{c)$\triangle$XYZ with $\angle{Y}=67\degree$, $\angle{Z}=90\degree$, and $y=21$ m} \\ \\
\begin{tabular}{l l l l}
    $\angle{X}=180\degree-90\degree-67\degree$ & $tan(23\degree)=\frac{x}{21}$ & $sin(23\degree)=\frac{8.9}{z}$
    & \\
    $\angle{B}= 23\degree$ & $x=21*tan(23\degree)$ & $z=\frac{8.9}{sin(75\degree)}$
    & \\
    & $x=8.9$ m & $z=22.8$ m
\end{tabular}

\textbf{d)$\triangle$PQR with $\angle{P}=90\degree$, $\angle{R}=51\degree$, and $q=150$ mm} \\ \\
\begin{tabular}{l l l l}
    $\angle{Q}=180\degree-90\degree-51\degree$ & $tan(51\degree)=\frac{r}{150}$ & $sin(51\degree)=\frac{185.2}{p}$ 
    & \\
    $\angle{Q}= 39\degree$ & $r=150*tan(51\degree)$ & $p=\frac{185.2}{sin(51\degree)}$
    & \\
    & $r=185.2$ mm & $p=238.3$ mm
\end{tabular}

\textbf{e)$\triangle$GHI with $\angle{I}=90\degree$, $g=1.5$ m, and $h=1.2$ m} \\ \\
\begin{tabular}{l l l l}
    $tanG=\frac{1.5}{1.2}$ & $\angle{H}=180\degree-90\degree-51\degree$ & $sin(39\degree)=\frac{1.2}{i}$
    & \\
    $\angle{G}=tan^{-1}(\frac{1.5}{1.2})$ & $\angle{H}=39\degree$ & $i=\frac{1.2}{sin(39\degree)}$
    & \\
    $\angle{G}=51\degree$ &  & $i=1.9$ m
\end{tabular}

\textbf{\emph{\textsc{3. Determine the measure of obtuse \angle{D} for each ratio.}}}
\begin{multicols}{3}
    \textbf{a)$sinD=0.45$} \\
    $\angle{D}=sin^{-1}(0.45)$ \\
    $\angle{D}=27\degree$ \\ \\
    \textbf{d)$sinD=0.60$} \\
    $\angle{D}=sinD^{-1}(0.60)$ \\
    $\angle{D}=37\degree$ \\
    \columnbreak
    
    \textbf{b)$cosD=-0.21$} \\
    $\angle{D}=cos^{-1}(-0.21)$ \\
    $\angle{D}=102\degree$ \\ \\
    \textbf{e)$sinD=-0.99$} \\
    $\angle{D}=sinD^{-1}(-0.99)$ \\
    $\angle{D}=172\degree$ \\
    \columnbreak
    
    \textbf{c)$tanD=-0.43$} \\
    $\angle{D}=sin^{-1}(-0.43)$ \\
    $\angle{D}=23\degree$ \\ \\
    \textbf{f)$sinD=0.60$} \\
    $\angle{D}=sinD^{-1}(0.60)$ \\
    $\angle{D}=37\degree$ \\
    \columnbreak    
\end{multicols}

\newpage

\textbf{\emph{\textsc{4. Decide whether you use the Sine Law or the Cosine Law to solve each triangle. Then, solve each triangle. (Refer to handout)}}}

\begin{tabular}{l l l l}
    \textbf{a)} Sine Law & &
    & \\
    $\angle{R}=180\degree-23\degree-22\degree$ & $\frac{r}{sin(135\degree)}=\frac{5.3}{sin(22\degree)}$ & $\frac{p}{sin(23\degree)}=\frac{5.3}{sin(22\degree)}$
    & \\
    $\angle{R}=135\degree$ & $r=\frac{5.3*sin(135\degree)}{sin(22\degree)}$ & $p=\frac{5.3*sin(23\degree)}{sin(22\degree)}$
    & \\
    & $r=10$ km & $p=5.5$ km
\end{tabular}

\begin{tabular}{l l l l}
    \textbf{b)} Cosine Law & &
    & \\
    $t=\sqrt{13^2+3^2-2(13)(3)*cos(32\degree)}$ & $cosV=\frac{10.6^2+13^2-3^2}{2(10.6)(13)}$ & $\angle{U}=180\degree-32\degree-9\degree$
    & \\
    $t=\sqrt{111.9}$ & $\angle{V}=cos^{-1}(0.99)$ & $\angle{U}=139\degree$
    & \\
    $t=10.6$ in & $\angle{V}=9\degree$ & 
    & \\
\end{tabular}

\textbf{\emph{\textsc{5. Determine $z$. (Refer to handout)}}}

\begin{tabular}{l l l l}
    $\angle{Y}=180\degree-83\degree-46\degree$ & $\frac{x}{sin(83\degree)}=\frac{9.9}{sin(51\degree)}$ & $z=\sqrt{12.6^2+14^2-2(12.6)(14)*cos(52\degree)}$
    & \\
    $\angle{Y}=51\degree$ & $x=\frac{9.9*sin(83\degree)}{sin(51\degree)}$ & $z=\sqrt{137.6}$
    & \\
    & $x=12.6$ ft & $z=11.7$ ft
    & \\
\end{tabular}

\textbf{\emph{\textsc{6. Two ships sail out from a harbour at the same time. One sails on a bearing of $015\degree$ and travels a distance of $32$ miles. The other ship sails $47$ miles on a bearing of $165\degree$. \\
a) How far apart are the ships? \\
b) What is the bearing from the first ship to the second ship?}}}

\begin{tabular}{l l l l}
    a) & $\angle{C}=(90\degree-15\degree)(90\degree-(180\degree-165\degree))$ & $c=\sqrt{47^2+32^2-2(47)(32)*cos(150\degree)}$
    & \\
    & $\angle{C}=150\degree$ & $c=\sqrt{5838}$
    & \\
    & & $c=76.4$ mi
    & \\
\end{tabular}
b) The bearing from the first ship to the second is 150\degree

\textbf{\emph{\textsc{7. a) Determine the area of this composite figure. All curves are quarter circles or semicircles. (Refer to handout) \\
b) Determine the surface area and volume of this composite object. (Refer to handout)}}}
\begin{tabular}{l l}
\end{tabular}

\begin{tabular}{l l l l}
    b) & $SA=2(4*4)+4(0.5*4)-2(\pi(0.7)^2)$ & $V=(4*4*0.5)-(\pi(0.7)^2(0.5))$
    & \\
    & $SA=40-3.1$ & $V=8-0.77$
    & \\
    & $SA=36.9$ cm$^2$ & $V=7.2$ cm$^3$
    & \\
\end{tabular}

\textbf{\emph{\textsc{8.For each perimeter, what are the dimensions of the rectangle with the maximum area? What is the area?}}} \\
\begin{tabular}{l l l l l}
    \textbf{a)28 m} & \textbf{b)44 in} & \textbf{c)10 cm} & \textbf{d)94 ft}
    & \\
\end{tabular} \\

\begin{tabular}{l l l l l}
    a)$A=(28/4)^2$ & b)$A=(44/4)^2$ & c)$A=(10/4)^2$ & d)$A=(94/4)^2$
    & \\
    $A=7^2$ & $A=11^2$ & $A=2.5^2$ & $A=23.5^2$
    & \\
    $A=49$ m$^2$ & $A=121$ in$^2$ & $A=6.3$ cm$^2$ & $A=552.3$ ft$^2$
    & \\
    $7x7$ & $11x11$ & $2.5x2.5$ & $23.5x23.5$
    & \\
\end{tabular}

\textbf{\emph{\textsc{9. Gizelle is designing an art project for children at her day care. She will have them use 10 paper clips to create a border for an art project. She is debating whether to use a rectangular or triangular border. \\
a) Each paper clip is 2 inches long. What are the side lengths of the rectangles and triangles she can construct? \\
b) What is the greatest rectangular or triangular area she can enclose? What shape does it have?}}} \\ \\
a) Rectangle: 1x4, 2x3 Triangle: 2x4x4, 4x3x3 \\
b) A 2x3 rectangle or a 4x3x3 isosceles triangle

\newpage

\textbf{\emph{\textsc{10. For each volume, what are the dimensions of the rectangular prism with the minimum surface area? What is the surface area?}}} \\
\begin{tabular}{l l l l l}
    \textbf{a)64 ft$^3$} & \textbf{b)729 m$^2$} & \textbf{c)225 cm$^2$} & \textbf{d)3000 in$^2$}
    & \\
\end{tabular} \\ \\
\begin{tabular}{l l l l l}
    a) $s=\sqrt[3]{64}$ & $SA=6*4^2$ & b) $s=\sqrt[3]{729}$ & $SA=6x9^2$
    & \\
    $s=4$ ft & $SA=6*16$ & $s=9$ m & $SA=6*81$
    & \\
    $4x4x4$ & $SA=96$ ft$^2$ & $9x9x9$ & $SA=486$ m$^2$
    & \\ \\
    c) $s=\sqrt[3]{225}$ & $SA=6*6.1^2$ & b) $s=\sqrt[3]{3000}$ & $SA=6x14.4^2$
    & \\
    $s=6.1$ cm & $SA=6*37.2$ & $s=14.4$ in & $SA=6*207.4$
    & \\
    $6.1x6.1x6.1$ & $SA=223.2$ cm$^2$ & $14.4x14.4x14.4$ & $SA=1244.4$ in$^2$
    & \\
\end{tabular}

\end{document}
