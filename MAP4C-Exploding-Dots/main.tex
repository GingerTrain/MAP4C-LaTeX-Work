\documentclass{article}
\usepackage[utf8]{inputenc}
\usepackage{parskip} % Removes indent
\usepackage{amssymb} % Includes therefore symbol
\usepackage{amsmath} % Fractions 

\title{MAP4C Exploding Dots}
\author{Eric Taylor}
\date{May 2017}

\begin{document}

\topmargin=0pt % Sets top margin to 0pt
\headheight=0pt % Sets header distance to 0pt
\headsep=0pt % Sets distance between header and body to 0pt
\footskip=0pt % Sets footer to 0pt
\textheight=700pt % Sets height of body to 700pt

\setlength{\parindent}{0pt} % Removes indent

\maketitle \thispagestyle{empty}

\textbf{\emph{\textsc{Question 1: \\
a) Verify that the code for 13 is 1101 in a $1 \leftarrow 2$ machine. \\
b) What is the code for 50?}}}

\begin{tabular}{l l}
    \textbf{a)} $1101 = 1(2^2)+1(2^2)+0(2)+1(1)$ 
    & \\
    $1101 = 13$
\end{tabular}

\begin{tabular}{l l l l l l l l}
    \textbf{b)} $50\%2=0$ & $25\%2=1$ & $12\%2=0$ & $6\%2=0$ & $3\%2=1$ & $1\%3=1$ &
    & \\
    $50/2=25$ & $(25-1)\%2=0$ & $12/2=6$ & $6/2=3$ & $(3-1)\%2=0$ & $1-1=0$
    & \\
    & $24/2=12$ & & & $2/2=1$ &
\end{tabular}
$\therefore$ the code for 50  in a $1 \leftarrow 2$ machine is 010011

\textbf{\emph{\textsc{Question 2: Could a number ever have 100211 as its code in a $1 \leftarrow 2$ machine?}}}

No, it is not possible for a number to ever have 100211 as its code in a $1 \leftarrow 2$ machine because the two dots in the 3rd position would explode, and produce a dot in the next position.

\textbf{\emph{\textsc{Question 3: Which number has code 10101 in a $1 \leftarrow 2$ machine?}}}

\begin{tabular}{l l}
    $10101 = 1(2^4)+0(2^3)+1(2^2)+0(2)+1(1)$ 
    & \\
    $10101 = 21$
\end{tabular}

\textbf{\emph{\textsc{Question 4: \\
a) Show that the $1 \leftarrow 3$ code for four is 11. \\
b) Show that the $1 \leftarrow 3$ code for twenty is 202}}}

\begin{tabular}{l l l}
    \textbf{a)} $11 = 1(3)+1(1)$ & \textbf{b)} $202 = 2(3^2)+0(3)+2(1)$
    & \\
    $11 = 4$ & $202=20$ 
\end{tabular}

\newpage

\textbf{\emph{\textsc{Question 5: \\
a) What is the $1 \leftarrow 3$ code for 13? \\
b) For 25?}}}

\begin{tabular}{l l l l}
    \textbf{a)} $13\%3=1$ & $4\%3=1$ & $1\%3=1$
    & \\
    $(13-1)/3=4$ & $(4-1)/3=1$ & $1-1=0$
\end{tabular}

$\therefore$ the code for 13 in a $1 \leftarrow 3$ machine is 111.

\begin{tabular}{l l l l}
    \textbf{b)} $25\%3=1$ & $8\%3=2$ & $2\%3=2$
    & \\
    $(25-1)/3=8$ & $(8-2)/3=2$ & $2-2=0$
\end{tabular}

$\therefore$ the code for 13 in a $1 \leftarrow 3$ machine is 221.

\textbf{\emph{\textsc{Question 6: Is it possible for a number to have a $1 \leftarrow 3$ code 2031? Explain}}}

No, it is not possible for a number to ever have 2031 as its code in a $1 \leftarrow 3$ machine because the three dots in the 2nd position would explode, and produce a dot in the next position. 

\textbf{\emph{\textsc{Question 7: What number has $1 \leftarrow 3$ code 1022?}}}

\begin{tabular}{l l}
    $1022=1(3^3)+0(3^2)+2(3)+2(1)$
    & \\
    $1022=35$
\end{tabular}

\textbf{\emph{\textsc{Question 8: What do you think the $1 \leftarrow 4$ rule is? What is the $1 \leftarrow 4$ code for the number thirteen?}}}

The $1 \leftarrow 4$ rule is, that whenever there are 4 dots in a box, the 4 dots explode and create 1 dot in the next box.

\begin{tabular}{l l l l l}
    $13\%4=1$ & $3\%4=3$
    & \\
    $(13-2)/4=3$ & $3-3=0$
\end{tabular}
$\therefore$ the code for 13 in a $1 \leftarrow 4$ machine is 31.

\textbf{\emph{\textsc{Question 9: What is the $1 \leftarrow 5$ code for the number thirteen?}}}

\begin{tabular}{l l l l l}
    $13\%5=3$ & $2\%5=2$
    & \\
    $(13-3)/5=2$ & $2-2=0$
\end{tabular}
$\therefore$ the code for 13 in a $1 \leftarrow 5$ machine is 23.

\textbf{\emph{\textsc{Question 10: What is the $1 \leftarrow 9$ code for the number thirteen?}}}

\begin{tabular}{l l l l l}
    $13\%9=4$ & $1\%9=1$
    & \\
    $(13-4)/9=1$ & $1-1=0$
\end{tabular}
$\therefore$ the code for 13 in a $1 \leftarrow 9$ machine is 14.

\textbf{\emph{\textsc{Question 11: What is the $1 \leftarrow 5$ code for the number twelve?}}}

\begin{tabular}{l l l l l}
    $12\%5=2$ & $2\%5=2$
    & \\
    $(12-2)/5=2$ & $2-2=0$
\end{tabular}
$\therefore$ the code for 12 in a $1 \leftarrow 5$ machine is 22.

\textbf{\emph{\textsc{Question 12: What is the $1 \leftarrow 9$ code for the number thirty?}}}

\begin{tabular}{l l l l l}
    $30\%9=3$ & $3\%9=3$
    & \\
    $(30-3)/9=3$ & $3-3=0$
\end{tabular}
$\therefore$ the code for 30 in a $1 \leftarrow 9$ machine is 33.

\newpage

\textbf{\emph{\textsc{Question 13: \\
a) What is the $1 \leftarrow 10$ code for the number thirteen? \\
b) What is the $1 \leftarrow 10$ code for the number thirty-seven? \\
c) What is the $1 \leftarrow 10$ code for the number 273? \\
d) What is the $1 \leftarrow 10$ code for the number 5846?}}}

\begin{tabular}{l l l}
    \textbf{a)} $13\%10=3$ & $1\%10=1$
    & \\
    $(13-3)/10=1$ & $1-1=0$
\end{tabular}
$\therefore$ the code for 13 in a $1 \leftarrow 10$ machine is 13.

\begin{tabular}{l l l}
    \textbf{b)} $37\%10=7$ & $3\%10=3$
    & \\
    $(37-7)/10=3$ & $3-3=0$
\end{tabular}
$\therefore$ the code for 37 in a $1 \leftarrow 10$ machine is 37.

\begin{tabular}{l l l l}
    \textbf{c)} $273\%10=3$ & $27\%10=7$ & $2\%10=2$
    & \\
    $(273-3)/10=27$ & $(27-7)/10=2$ & $2-2=0$
\end{tabular}

$\therefore$ the code for 273 in a $1 \leftarrow 10$ machine is 273.

\begin{tabular}{l l l l l}
    \textbf{d)} $5846\%10=6$ & $584\%10=4$ & $58\%10=8$ & $5\%10=5$
    & \\
    $(5846-6)/10=584$ & $(584-4)/10=58$ & $(58-8)/10=5$ & $5-5=0$
\end{tabular}

$\therefore$ the code for 5846 in a $1 \leftarrow 10$ machine is 5846.

\textbf{\emph{\textsc{Question 14: What are the values of the boxes a few more places to the left?}}}

16, 32, 64, 128, 256, 512, 1024...

\textbf{\emph{\textsc{Question 15: What number has $1 \leftarrow 2$ code 100101?}}}

$100101=1(2^5)+0(2^4)+0(2^3)+1(2^2)+0(2)+1(1)$ \\
$100101=37$

\textbf{\emph{\textsc{Question 16: What is the $1 \leftarrow 2$ code for the number two hundred?}}}


\begin{tabular}{l l l l l l l l l}
    $200\%2=0$ & $100\%2=0$ & $50\%2=0$ & $25\%2=1$ 
    & \\
    $200/2=100$ & $100/2=50$ & $50/2=25$ & $(25-1)/2=12$
    & \\ \\
    $12\%2=0$ & $6\%2=0$ & $3\%2=1$ & $1\%2=1$ 
    & \\
    $12/2=6$ & $6/2=3$ & $(3-1)/2=1$ & $1-1=0$
\end{tabular}

$\therefore$ the code for 200 in a $1 \leftarrow 2$ machine is 11001000.

\newpage

\textbf{\emph{\textsc{Question 17: \\
a) What is the next number? \\
b) Could we say that the $1 \leftarrow 3$ code for fifteen was 0120? That is, is it okay to put zeros in the front of these codes? What about zeros at the ends of codes? Are they optional? Is it okay to leave off the last zero of the code 120 for fifteen and just write instead 12? \\
c) What number has $1 \leftarrow 3$ code 21002 \\
d) What is the $1 \leftarrow 3$ code for two hundred?}}}

\textbf{a)} 81

\textbf{b)} Yes, but it is not recommended. Considering the value is unused in both the calculations and the final answer, it is unnecessary to include it in the answer. It is okay to put zeros in the front of the codes, however it isn't recommended for the same reasons stated previously. The zeros at the ends of codes are very important and mandatory. The rightmost value of the code is assumed to be the starting point, so if you were to take out the zeros at the ends of the codes then all the values are shifted over one position, messing up the values and codes. Using the code 120 for a $1 \leftarrow 3$ machine as an example, if you were to take off the zero at the end and turn it into 12, the value would change from 15 to 5

\textbf{c)} $21002=2(3^4)+1(3^3)+0(3^2)+0(3)+2(1)$ \\
$21002=191$

\begin{tabular}{l l l l l l}
    \textbf{d)} $200\%3=2$ & $66\%3=0$ & $22\%3=1$ & $7\%3=1$ & $2\%=2$
    & \\
    $(200-2)/3=66$ & $66/3=22$ & $(22-1)/3=7$ & $(7-1)/3=2$ & $2-2=0$
\end{tabular}

$\therefore$ the code for 200 in a $1 \leftarrow 3$ machine is 21102.

\textbf{\emph{\textsc{Question 18: What is the value of each box? \\
a) What is the $1 \leftarrow 4$ code for 29? \\
b) What number has $1 \leftarrow 4$ code 132?}}}

The value of each box is: 1, 4, 16, 64, 256

\begin{tabular}{l l l l}
    \textbf{a)} $29\%4=1$ & $7\%4=3$ & $1\%4=1$
    & \\
    $(29-1)/4=7$ & $(7-3)/4=1$ & $1-1=0$
\end{tabular}

$\therefore$ the code for 29 in a $1 \leftarrow 4$ machine is 131.

\textbf{b)} $132=1(4^2)+3(4)+2(1)$ \\
$132=30$

\newpage

\textbf{\emph{\textsc{Question 19: \\
a) What is the value of each box? \\
b) What is the code for the number 98723 in a $1 \leftarrow 10$ system? \\
c) When we write the number 7842 the 7 is represents what quantity? The 4 is four groups of what value? The 8 is eight groups of what value? The 2 is two groups of what value? \\
d) Why do human beings like the $1 \leftarrow 10$ system for writing numbers? Why the number 10? What do we like to use on the human body for counting? Would Martians likely use the $1 \leftarrow 10$ system for their mathematics, why or why not?}}}

\textbf{a)} The value of each box is: 1, 10, 100, 1000, 10000

\begin{tabular}{l l l l l l}
    \textbf{b)} $98723\%10=3$ & $9872\%10=2$ & $987\%10=7$ 
    & \\ 
    $(98723-3)/10=9872$ & $(9872-2)/10=987$ & $(987-7)/10=98$ 
    & \\ \\
    $98\%10=8$ & $9\%10=9$
    & \\
    $(98-8)/10=9$ & $9-9=0$
\end{tabular}

$\therefore$ the code for 98723 in a $1 \leftarrow 10$ machine is 98723.

\textbf{c)} The 7 represents the quantity of 1000's. The 4 is four groups of 10. The 8 is eight groups of 100. The 2 is two groups of 1.

\textbf{d)} Humans use the $1 \leftarrow 10$ system for writing numbers because it is easy to calculate in your head, and it puts the numbers in nice groups. The number 10 because doing math with base 10 is easy to calculate. Humans like to use their fingers for counting things. Martians probably wouldn't use the $1 \leftarrow 10$ system because they probably have a different numerical system, and they also probably don't have 10 fingers... they might not even have arms.

\textbf{\emph{\textsc{Question 20: What is $3704+2214$?}}}

$3704+2214=5918$

\textbf{\emph{\textsc{Question 21: Solve the following problems thinking about dots and boxes, then translate the answer into something the rest of the world can understand. \\
a) $148+323$ \\
b) $567+271$ \\
c) $377+188$ \\
d) $582+714$ \\
e) $310462872+389107123$ \\
f) $87263716381+18778274824$}}}

\begin{tabular}{l l l l l}
    \textbf{a)} $148+323=4\vline6\vline11$ & \textbf{b)} $567+271=7\vline13\vline8$ & \textbf{c)} $377+188=4\vline15\vline15$ & \textbf{d)} $582+714=12\vline9\vline6$
    & \\
    $4\vline6\vline11=471$ & $7\vline13\vline8=838$ & $4\vline15\vline15=565$ & $12\vline9\vline6\vline=1296$
    & \\
\end{tabular}

\begin{tabular}{l l l}
    \textbf{e)} $310462872+389107123=6\vline9\vline9\vline5\vline6\vline9\vline9\vline9\vline5$ & \textbf{f)} $87263716381+18778274824=9\vline15\vline9\vline13\vline11\vline9\vline8\vline10\vline11\vline10\vline5$
    & \\
    $6\vline9\vline9\vline5\vline6\vline9\vline9\vline9\vline5=699569995$ & $9\vline15\vline9\vline13\vline11\vline9\vline8\vline10\vline11\vline10\vline5=106041991205$
\end{tabular}

\textbf{\emph{\textsc{Question 23: Here is an addition problem in a $1 \leftarrow 5$ system: $20413+13244$ \\
a) What is the $1 \leftarrow 5$ answer? \\
b) If this were an addition problem in a $1 \leftarrow 10$ system, what would the answer be?}}}

\begin{tabular}{l l l}
    \textbf{a)} $20413+13244=3\vline3\vline6\vline5\vline7\vline$ & \textbf{b)} $20413+13244=3\vline3\vline6\vline5\vline7\vline$ 
    & \\
    $3\vline3\vline6\vline5\vline7\vline=34212$ & $3\vline3\vline6\vline5\vline7\vline=33657$
\end{tabular}

\newpage

\textbf{\emph{\textsc{Question 24: Jenny was asked to compute $243192*4$. She wrote $243192*4=8\vline16\vline12\vline4\vline36\vline8$. \\
a) What was she thinking? Why is this a mathematically correct answer? \\
b) Translate the answer into a number that the rest of the world can understand. \\
c) Find the answers to these multiplication problems: \\
i) $156*3$ ii) $2873*2$ iii) $71181*5$ iv) $3726510392*2$ v) $765*9$ }}}

\textbf{a)} Jenny was thinking that multiplying each digit by 4 would produce the same result as multiplying the whole number by 4. This is a mathematically correct answer because if you were to add up each of the individual multiplications, you would get the same answer as if you were to multiply the whole number by 4.

\textbf{b)} $800000+160000+12000+400+36+8=972768$

\begin{tabular}{l l l l}
   \textbf{c)} $156*3=3\vline15\vline18$ & $2873*2=4\vline16\vline14\vline6$ & $71181*5=35\vline5\vline40\vline5$
   & \\
   $3\vline15\vline18=468$ & $4\vline16\vline14\vline6=5746$ & $35\vline5\vline40\vline=5355905$
   & \\ \\
   $3726510392*2=6\vline14\vline4\vline12\vline10\vline2\vline0\vline6\vline18\vline4$ & $765*9=63\vline54\vline45$ &
   & \\
   $6\vline14\vline4\vline12\vline10\vline2\vline0\vline6\vline18\vline4=7453020784$ & $63\vline54\vline45=6885$ &
\end{tabular}











\end{document}
